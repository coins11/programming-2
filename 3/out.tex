\documentclass[11pt,a4paper,a4j]{jsarticle}
\usepackage{okumacro, fancybox, ascmac}

\title{プログラミング入門II 演習報告書}
\author{吉村 優(201111411)\\
mail:yoshimura\_yuu@coins.tsukuba.ac.jp}

\begin{document}
\maketitle

\part*{演習3}
\section{}
\subsection{ソース}
\begin{shadebox}
\begin{verbatim}
#include <stdio.h>

int main(int argc, char *argv[])
{
    int count;
    FILE *fp;

    if (argc != 2) {
        printf("missing file argument\n");
        return 1;
    }

    if ((fp = fopen(argv[1], "r")) == NULL) {
        printf("can't open %s\n", argv[1]);
        return 1;
    }

    count = 0;
    while (fgetc(fp) != EOF) {
        count++;
    }

    printf("%d letters\n", count);
    fclose(fp);

    return 0;
}
\end{verbatim}
\end{shadebox}

\subsection{結果}
\begin{shadebox}
\begin{verbatim}
yoshimura_yuu $ ./1 1.c
362 letters
\end{verbatim}
\end{shadebox}

\section{}
\subsection{ソース}
\begin{shadebox}
\begin{verbatim}
#include <stdio.h>

int main(int argc, char *argv[])
{
    int ch;
    FILE *sfp, *dfp;

    if (argc != 3) {
        printf("missing file argument\n");
        return 1;
    }

    if ((sfp = fopen(argv[1], "r")) == NULL) {
        printf("can't open %s\n", argv[1]);
        return 1;
    }

    if ((dfp = fopen(argv[2], "w")) == NULL) {
        printf("can't open %s\n", argv[2]);
        fclose(sfp);
        return 1;
    }

    while ((ch = fgetc(sfp)) != EOF) {
        if (97 <= ch && ch <= 122) { 
            fputc(ch-32, dfp);
        } else {
            fputc(ch, dfp);
        }
    }

    fclose(dfp);
    fclose(sfp);

    return 0;
}
\end{verbatim}
\end{shadebox}

\subsection{結果}
\begin{shadebox}
\begin{verbatim}
[11] [~/tmp/pro2/3]
yoshimura_yuu $ ./2 1.c copy.txt

------------ copy.txt -------------
#INCLUDE <STDIO.H>

INT MAIN(INT ARGC, CHAR *ARGV[])
{
    INT COUNT;
    FILE *FP;

    IF (ARGC != 2) {
        PRINTF("MISSING FILE ARGUMENT\N");
        RETURN 1;
    }

    IF ((FP = FOPEN(ARGV[1], "R")) == NULL) {
        PRINTF("CAN'T OPEN %S\N", ARGV[1]);
        RETURN 1;
    }

    COUNT = 0;
    WHILE (FGETC(FP) != EOF) {
        COUNT++;
    }

    PRINTF("%D LETTERS\N", COUNT);
    FCLOSE(FP);

    RETURN 0;
}
\end{verbatim}
\end{shadebox}

\section{}
\subsection{ソース}
\begin{verbatim}
#include <stdio.h>

struct person {
    char name[256];
    int  score;
};

int main(int argc, char *argv[]) {
    int n, i, j;
    FILE *fp;
    struct person p[128];
    struct person tmp;

    if (argc != 2) {
        printf("missing file argument\n");
        return 1;
    }

    if ((fp = fopen(argv[1], "r")) == NULL) {
        printf("can't open %s\n", argv[1]);
        return 1;
    }


    for (i=0; fscanf(fp, "%s\t\t%d", p[i].name, &p[i].score) != EOF; i++)
    
    n = i;
    
    fclose(fp);

    for (i=0; i<n; i++) {
        for (j=n; j>i; j--) {
            if (p[j-1].score > p[j].score) {
                tmp    = p[j];
                p[j]   = p[j-1];
                p[j-1] = tmp;
            }
        }
    }
    
    for (i=0; i<n; i++) {
        printf("%s\t%d\n", p[i].name, p[i].score);
    }    

    return 0;
}
\end{verbatim}

\subsection{結果}
\begin{shadebox}
\begin{verbatim}
yoshimura_yuu $ ./3 prog2-ex3-data.txt
Kojima  10
Kuno    10
Miura   12
Ojima   30
Yamada  33
Nagai   35
Kudou   40
Kawasaki        40
Kameda  44
Sato    50
Takahasi        60
Suzuki  60
Suda    65
Hayashi 66
Kato    70
Matumoto        70
Akiyama 80
Kimura  85
Nakamura        95
\end{verbatim}
\end{shadebox}

\end{document}
